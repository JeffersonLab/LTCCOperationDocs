\section{Low Threshold Cherenkov Counter}

The CLAS12 Low Threshold Cherenkov Counter (LTCC) is composed of six identical sectors. The sectors are filled 
with $C_4F_{10}$ gas supplied by the Hall-B gas system. The gas is cleaned, re-circulated and maintained at a pressure between 1-4" of water
with gas flow controllers and bubble pressure relief units.
Each sector contains 36 PMTs energized by a HV power supply. Each PMT produces two outputs, 
connected to VME electronics (FADCs, TDCs) in the Forward Carriage.


\subsection{Hazards} 

There are three hazards identified with operation of the LTCC system. 
\begin{itemize}
\item Electrical hazard when the HVPS is energized for the PMTs.
\item Fall hazards from using man-lifts or ladders to access system elements during maintenance and testing operations. 
\item Gas pressure hazards when the detector is pressurized with C4F10, typically 1-4 inches of water pressure.
\end{itemize}

\subsection{Mitigations}

The HV hazard is mitigated by a maximum current settings on the power supply.

Harness training, man-lift training, ladder training, fall protection training provides mitigation for the fall hazard during the detector maintenance.

Detectors pressure and vacuum is limited to a maximum of 4"wc by the Bubbler pressure relief units.


\subsection{Responsible personnel}

Individuals responsible for the system are:

\begin{table}[!htb]
 \centering
 \begin{tabular}{|c|c|c|c|c|}
\hline
 Name&Dept.&Phone&email&Comments \\ \hline
 Expert on call& &&& 1st contact \\ \hline
M. Ungaro&JLAB&757-269-7578&\href{mailto:ungaro@jlab.org}{\nolinkurl{ungaro@jlab.org}}&Contact \\ \hline
 & &&\href{mailto:}{\nolinkurl{}}& Contact  \\ \hline
 \end{tabular}
\caption{Personnel responsible for the CLAS12 Low Threshold Cherenkov Counter.} 
\label{tb:ltcc}
\end{table}

